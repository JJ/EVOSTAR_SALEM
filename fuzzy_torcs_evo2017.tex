\documentclass[runningheads,a4paper]{llncs}

\usepackage[latin1]{inputenc}
\usepackage{graphicx,color,url}
\usepackage[dvips]{epsfig}
\usepackage{verbatim}



\newcommand{\keywords}[1]{\par\addvspace\baselineskip
\noindent\keywordname\enspace\ignorespaces#1}

\providecommand{\tabularnewline}{\\}

\begin{document}

\mainmatter  % start of an individual contribution

% first the title is needed
\title{Fuzzy Controller for TORCS}
% Antonio - T�tulo provisional, actualizar con el definitivo


% a short form should be given in case it is too long for the running head
\titlerunning{There can be only one}

% the name(s) of the author(s) follow(s) next
%
% NB: Chinese authors should write their first names(s) in front of
% their surnames. This ensures that the names appear correctly in
% the running heads and the author index.
%

%\author{M. Salem \and A.M Mora \and \and J. J. Merelo \inst{1}}
\author{A. N. Onymous\inst{1}%
\thanks{No Institute}}
% Antonio - Los autores que participen
%
\authorrunning{Anonymous, A}
% (feature abused for this document to repeat the title also on left hand pages)

% the affiliations are given next; don't give your e-mail address
% unless you accept that it will be published
%\institute{Dept. of Computer Architecture and Technology, University of Granada, Spain}
\institute{Anonymous Institute}


%
% NB: a more complex sample for affiliations and the mapping to the
% corresponding authors can be found in the file "llncs.dem"
% (search for the string "\mainmatter" where a contribution starts).
% "llncs.dem" accompanies the document class "llncs.cls".
%

\maketitle

%
%%%%%%%%%%%%%%%%%%%%%%%%%%%%%%%   ABSTRACT   %%%%%%%%%%%%%%%%%%%%%%%%%%%%%%%
%
\begin{abstract}
The abstract
%
%\keywords{Videogames, Fuzzy Controller, TORCS}
\end{abstract}

%
%%%%%%%%%%%%%%%%%%%%%%%%%%%%%%%   INTRODUCTION   %%%%%%%%%%%%%%%%%%%%%%%%%%%%%%%
%
\section{Introduction}
\label{sec:intro}



%%%%%%%%%%%%%%%%%%%%%%%%%%%%%%  STATE OF THE ART  %%%%%%%%%%%%%%%%%%%%%%%%%%%%%
%
\section{Background and State of the Art}
\label{subsec:soa}

A very good controller for TORCS is \cite{CarRacing_Pelta09}...

%%%%%%%%%%%%%%%%%%%%%%%%%%%%%%  TORCS  %%%%%%%%%%%%%%%%%%%%%%%%%%%%%
%
\section{Problem Description: TORCS}
\label{subsec:torcs}

TORCS \cite{WebTORCS} is an Open Racing Car Simulator...


%%%%%%%%%%%%%%%%%%%%%%%%%%%  FUZZY CONTROLLER  %%%%%%%%%%%%%%%%%%%%%%%%%%%%
%
\section{Proposed Fuzzy Controller}
\label{subsec:fuzzy_controller}


%%%%%%%%%%%%%%%%%%%%%%%%%%%%%%  RESULTS  %%%%%%%%%%%%%%%%%%%%%%%%%%%%%
%
\section{Simulation Results}
\label{subsec:results}


%%%%%%%%%%%%%%%%%%%%%%%%%%%%%  CONCLUSIONS  %%%%%%%%%%%%%%%%%%%%%%%%%%%%
%
\section{Conclusions and Future Work}
\label{subsec:conclusions}




%%%%%%%%%%%%%%%%%%%%%%%%%%%%%%%%%%%%%%%%%%%%%%%%%%%%%%%%%%%%%%%%%%%%%%%

\section*{Acknowledgments}

Omitted for Blind reviews.

%This work has been supported in part by projects 
%EPHEMECH (TIN2014-56494-C4-3-P, Spanish Ministerio de Econom�a y Competitividad), 
%PROY-PP2015-06 (Plan Propio 2015 UGR), 
%PETRA (SPIP2014-01437, funded by Direcci�n General de Tr�fico),
%CEI2015-MP-V17 (awarded by CEI BioTIC Granada), and 
%PRY142/14 (funded by Fundaci�n P�blica Andaluza Centro de Estudios Andaluces en la IX Convocatoria de Proyectos de Investigaci�n).


\bibliographystyle{splncs03}
\bibliography{fuzzy_torcs}



\end{document}
